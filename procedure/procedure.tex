\documentclass[a4paper]{article}

\usepackage{graphicx}
\usepackage{natbib}
\usepackage{amsmath}
\usepackage[utf8]{inputenc}
\usepackage[portuguese]{babel}
\usepackage[left=2cm,right=2cm,top=2cm]{geometry}
\usepackage{multicol}
\usepackage{caption}
\setlength\parindent{0pt}

\renewcommand{\labelenumi}{\alph{enumi}.}
\newcommand{\igo}{\textit{iGo}}
\newcommand{\comment}[1]{}

\title{\igo\\Guião para Testes com Utilizadores\\\small Interfaces Pessoa Máquina}
\author{Baltasar Dinis 89416, Afonso Ribeiro 86752, Francisco Figueiredo 89443}

\date{\today}
\begin{document}
\maketitle

\begin{abstract}
  Este guião visa orientar os testes de utilizador do \igo,
  implementado pelo grupo 19. Será necessário um computador para
  a realização do teste. O local deverá ser calmo, para que o
  utilizador consiga ouvir com clareza quem está a administrar o
  teste. Procura-se que as condições sejam semelhantes entre os
  vários testes.

  Anexos encontram-se os questionários demográfico e de
  usabilidade.
\end{abstract}

%\begin{multicols}{2}
  \section{Introdução}

  Bem vindo ao \igo!

  Nesta aplicação para smartwatch --- desenvolvida no âmbito da
  unidade curricular de Interfaces Pessoa Máquina --- os
  utilizadores têm no seu pulso o necessário para tornarem as suas
  férias espectaculares!

  Com este teste procuramos avaliar a usabilidade da aplicação,
  de forma a obter dados que permitam o seu melhoramento em
  iterações futuras.

  Será pedido que sejam cumpridas três tarefas, e serão
  contabilizados três parâmetros:
  \begin{enumerate}
    \item
      Tempo necessário;
    \item
      Número de cliques;
    \item
      Número de erros.
  \end{enumerate}

  Para o teste ser iniciado, o utilizador deve preencher o
  questionário com os seus dados e o formulário de consentimento.
  No final, será preenchido ainda um questionário de satisfação.

  Poderão ser dadas sugestões ou opiniões, mas não podem ser
  colocadas questões. O questionado não poderá interagir com o
  observador e está livre de abandonar o teste em qualquer
  altura.

  Salienta-se que os testes têm como objetivo avaliar a
  \textbf{interface} e não o utilizador e que os dados
  recolhidos são anónimos e têm como único objetivo melhorar o
  sistema.

  Agradecemos desde já a colaboração.

  \section{Avaliação}

  \subsection{Visitar um ponto de interesse}
  \textbf{Tarefa}: Encontrar e seguir o caminho para o Museu
  Calouste Gulbenkian.

  \textbf{Eficácia}\\
  \textit{Critério de Mensurabilidade}: Número de cliques e
  número de erros;

  \textit{Execução Esperada}: É expectável que o utilizador
  realize esta tarefa com no máximo 0 erros e 8 cliques.
  Sendo que 60\% realizá com 0 erros e 6 cliques.

  \textbf{Efeciência}\\
  \textit{Critério de Mensurabilidade}: Tempo necessário para
  realizar a tarefa;

  \textit{Execução Esperada}: O tempo médio para completar esta
  tarefa será inferior a 30 segundos. Todos os utilizadores levarão
  menos de 1 minuto a realizar a tarefa.

  \textbf{Satisfação}\\
  \textit{Medida}: Escala de Likert (5 niveis)
  1: Foi fácil encontrar o mapa;
  2: Foi fácil encontrar o Museu Calouste Gulbenkian;
  3: O mapa foi intuitivo, e seria útil caso procurasse o museu;

  \textit{Esperada}: 90\% dos utilizadores deverá optar por selecionar
  4 ou superior em todas as escalas.

  \subsection{Tirar uma Foto com um Dispositivo Externo}
  \textbf{Tarefa}: Tirar uma foto num dispositivo externo,
  colocando um temporizador de 5 segundos e aumentando o brilho.

  \textbf{Eficácia}\\
  \textit{Critério de Mensurabilidade}: Número de cliques e
  número de erros;

  \textit{Execução Esperada}: É expectável que o utilizador
  realize esta tarefa com no máximo 1 erros e 12 cliques.
  Sendo que 75\% realizá com 0 erros e 9 cliques.

  \textbf{Efeciência}\\
  \textit{Critério de Mensurabilidade}: Tempo necessário para
  realizar a tarefa;

  \textit{Execução Esperada}: O tempo médio para completar esta
  tarefa será inferior a 45 segundos. Todos os utilizadores levarão
  menos de 1 minuto e 20 segundos a realizar a tarefa.

  \textbf{Satisfação}\\
  \textit{Medida}: Escala de Likert (5 niveis)
  1: Foi fácil sincronizar com o dispositivo externo;
  2: Foi fácil ajustar o temporizador;
  3: Foi fácil ajustar o brilho;
  4: Foi fácil tirar a fotografia

  \textit{Esperada}: 95\% dos utilizadores deverá optar por selecionar 4 ou superior em todas as escalas. Sendo que
  65\% destes opte por selecionar 5;

  \subsection{Partilhar uma Fotografia}
  \textbf{Tarefa}: Partilhar duas fotos à
  escolha no myWeb, adicionando a localização e uma descrição,
  também à escolha.

  \textbf{Eficácia}\\
  \textit{Critério de Mensurabilidade}: Número de cliques e
  número de erros;

  \textit{Execução Esperada}: É expectável que o utilizador
  realize esta tarefa com no máximo 1 erros e 12 cliques.
  Espera-se que 20\% leve menos de 20 cliques pois não
  percebeu que podia partilhar 2 fotografias ao memsmo tempo.

  \textbf{Efeciência}\\
  \textit{Critério de Mensurabilidade}: Tempo necessário para
  realizar a tarefa;

  \textit{Execução Esperada}: O tempo médio para completar esta
  tarefa será inferior a 50 segundos. Todos os utilizadores levarão
  menos de 1 minuto e meio a realizar a tarefa.

  \textbf{Satisfação}\\
  \textit{Medida}: Escala de Likert (5 niveis)
  1: Foi fácil navegar até à galeria;
  2: Foi fácil selecionar as fotografias;
  3: Foi fácil adicionar a localização;
  4: Foi fácil partilhar a foto no myWeb

  \textit{Esperada}: 90\% dos utilizadores deverá optar por selecionar
  4 ou superior em todas as escalas excepto na segunda.
  Onde se espera que 85\% selecion 3 ou superior.

%\end{multicols}

  \section{Balanço}

  No final do teste, os utilizadores deverão responder a
  perguntas sobre a sua satisfação sobre as funcionalidades e a
  realização das tarefas.

  \section{Caracterização dos utilizadores}

  

  \section{Análise estatística}



  \section{Conclusão}

  


\end{document}
